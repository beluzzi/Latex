%%%%%%%%%%%%%%%%%%%%%%%%%%%%%%%%%%%%%%%%%%%%%%%%%%%%%%%%%%% Custom Image resize command -> requires \usepackage{graphix, calc}

\newlength{\textBoxHeight}
\newlength{\imageHeight}

\newcommand{\textAndImage}[2]{
	
	\settototalheight{\textBoxHeight}{\vbox{#1}}
	#1
	\setlength{\imageHeight}{\textheight-(\textBoxHeight+25pt)} 
  
  %The +25pt is a safety buffer, you can make it smaller and see at what point the first picture breaks
	
  \vfill
	\begin{center}
			\includegraphics[width=\linewidth,height=\imageHeight,keepaspectratio=true]{#2}
	\end{center}
	\vfill
}

%Usage: \textAndImage{Text above the image}{Path to image}

%%%%%%%%%%%%%%%%%%%%%%%%%%%%%%%%%%%%%%%%%%%%%%%%%%%%%%%%%%% Custom Image resize command with captions -> requires \usepackage{graphix, calc}

\newlength{\textBoxHeight}
\newlength{\imageHeight}

\newcommand{\textAndImage}[4]{
	
	\settototalheight{\textBoxHeight}{\vbox{#1}}
	#1
	\setlength{\imageHeight}{\textheight-(\textBoxHeight+75pt)} 
  
  %The +75pt is a safety buffer for captions, you can make this smaller if you dont use figures
	
  \vfill
	\begin{center}
		\begin{figure}[h!]
			\includegraphics[width=\linewidth,height=\imageHeight,keepaspectratio=true]{#2}
			\caption{#3}
			\label{#4}
		\end{figure}
	\end{center}
	\vfill
}

%Usage: \textAndImage{Text above the image}{Path to image}{Caption}{figure label e.g. fig:xx}

%%%%%%%%%%%%%%%%%%%%%%%%%%%%%%%%%%%%%%%%%%%%%%%%%%%%%%%%%%%
